%*******************************************************
% Abstract
%*******************************************************
%\renewcommand{\abstractname}{Abstract}
\pdfbookmark[1]{Abstract}{Abstract}
\begingroup
\let\clearpage\relax
\let\cleardoublepage\relax
\let\cleardoublepage\relax

\chapter*{Abstract}

Areas exposed to events or processes such as armed conflict, natural disasters or even climate change generally lack up-to-date, accessible data due to a variety of barriers (e.g. security, politics, biased providers). Free and open-access Earth observation (EO) data are being increasingly generated with global coverage at higher spatial resolutions and temporal frequencies than ever before. High spatial resolution (5-30m) imagery enables monitoring of large-scale areas beneficial for monitoring such events or processes. This data requires automated workflows for handling, processing and analysis, including methods to convert data into valid information.

Indicator extraction is one way to translate this data into meaningful information.

because the spatial resolution does not allow direct measurements of most objects on Earth (i.e. mixed pixels).

Improved situation assessment of areas where barriers to in-field data collection exist may be achieved by developing suitable indicators. In the case of events or processes that affect human populations, indicators of livelihood stability are vital to decision-makers. Due to the global and consistent nature of EO-data, they are ideal candidates for use in crisis indicators, combined or integrated with additional non-EO-data sources.  For example, night-time light EO-data integrated with the Joint Research Center’s Global Human Settlement Layer (GHSL) was used to estimate the size and location of the affected population in Syria. Such crisis indicators are envisioned to provide evidence-based knowledge to support crisis monitoring and impact assessment, but very few have been developed, applied and validated to date.

This study reviews the state-of-the-art of existing and envisioned humanitarian crisis indicators utilizing EO-data with a focus on solutions concerned with livelihood security (e.g. changes in agricultural areas, droughts, floods, power shortages). Emphasis in this review is given to indicators that monitor larger-scale areas, apply semi-automated to fully-automated workflows and utilise Sentinel-1/2/3 data. The aim is to provide an overview of the current state of development of EO-based humanitarian crisis indicators, and to distill knowledge about the contributions high resolution multi-spectral images (e.g. Sentinel-2) may offer. Ideally, future developments in this field will offer more solutions that are integrated or combined with at least one non-EO data source and utilise increasingly automated workflows.

A proof-of-concept implementation of a generic, semantic EO data cube with automated daily integration and semantic enrichment of Sentinel-2 data is presented and applied to Syria, a country with low annual cloud cover percentages beneficial for surface analysis using multi-spectral data. Using this implementation, surface water dynamics can be queried and analysed, and changes or losses to irrigated agricultural land over time are detected as a suggested indicator for instability. Challenges for analysis are posed not only by the identification of significant crisis indicators, requiring a combination of deductive and inductive methods, but also by the development of large-scale, automated (repeatable and reliable) methods for extracting indicators from relatively unwieldy big EO data. Analysis is applied along the entire Western and Northern borders of Syria, utilizing Sentinel-2 data, with a focus on the automation of information extraction and integration of derived information with at least one additional data source in a semi-automatic workflow. The idea is that such application-independent, semantic data cubes can facilitate reproducible and repeatable monitoring of land cover changes and the development of transferable, generic EO-based indicators to support international initiatives, such as the United Nations’ Sustainable Development Goals.

\emph{Keywords:} remote sensing, big Earth data, data cube, semantic enrichment, reproducible research, crisis indicators, livelihood security


\vfill

\begin{otherlanguage}{ngerman}
\pdfbookmark[1]{Zusammenfassung}{Zusammenfassung}
\chapter*{Zusammenfassung}

Kurze Zusammenfassung des Inhaltes in deutscher Sprache\dots

\emph{Schlagwörter:}

\end{otherlanguage}

\endgroup

\vfill
