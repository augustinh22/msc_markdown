%*******************************************************
% Abstract
%*******************************************************
%\renewcommand{\abstractname}{Abstract}
\cleardoublepage
\pdfbookmark[1]{Abstract}{Abstract}
\begingroup
%\let\clearpage\relax
\let\cleardoublepage\relax
\let\cleardoublepage\relax

\chapter*{Abstract}

Free and open-access \acf{EO} data with global coverage are being increasingly generated at higher spatial resolutions and temporal frequencies than ever before. Imagery with high spatial resolution (5-30\acs{m}) enables multi-temporal and persistent monitoring of large spatial extents for various applications. It does this in an unbiased, consistent way, independent of political borders. Free and open \acs{EO} is a growing collection of data that is one of the few globally consistent sources available for generating information in support of international initiatives. However, this data requires automated workflows for handling, processing and analysis, including methods to convert data into valid information. Because such optical \acs{EO} data does not allow direct measurements of most objects, processes or events on Earth, indicator extraction is one way to translate this data into meaningful information.

This thesis presents an overview of the current state of open and free \acs{EO} data in a larger framework of international initiatives, such as the \acs{UN} \acp{SDG}, big Earth data and data cube technologies. Based on this information, a proof-of-concept implementation of a semantic \acs{EO} data cube was created that is automatically updated with the newest Sentinel-2 data on a daily basis, including generic semantic enrichment. It continuously incorporates all of the available data for Sentinel-2 granules located in north-western Syria, a country characterised by low average monthly cloud cover throughout the year, making it an ideal location for surface analysis using multi-spectral data. The output of semantic queries based on vegetation- and water-like semi-concepts from the generic semantic information layer are presented, including challenges faced in their interpretation. These outputs could be seen as documenting surface water and vegetation dynamics over time. Their intended purpose is to move in the direction of generating ad-hoc reproducible, repeatable and spatially-explicit information as indicators.

\emph{Keywords:} remote sensing | big Earth data | data cube | semantic enrichment | reproducible research | sustainable development goals


\vfill



\begin{otherlanguage}{ngerman}
\clearpage
\pdfbookmark[1]{Zusammenfassung}{Zusammenfassung}
\chapter*{Zusammenfassung}

Kostenlose und frei zugängliche \acs{EO}-Daten mit globaler Abdeckung werden zunehmend bei höheren räumlichen Auflösungen und zeitlichen Häufigkeiten als je zuvor erzeugt. Bilder mit hoher räumlicher Auflösung (5-30\acs{m}) ermöglichen die multi-temporale und dauerhafte Überwachung großer räumlicher Ausdehnung für verschiedene Anwendungen. Dies geschieht auf eine unvoreingenommene, konsistente Weise, unabhängig von politischen Grenzen. Solche \acs{EO} besteht aus eine wachsende Sammlung von Daten, die eine der wenigen konstante Quellen mit globaler Abdeckung ist, die zur Generierung von Informationen um internationaler Initiativen zu unterstützen verwendet werden kann. Diese Daten erfordern jedoch automatisierte Workflows zur Handhabung, Verarbeitung und Analyse, einschließlich Methoden um Daten in Informationen umzuwandeln. Da solche optischen \acs{EO}-Daten keine direkten Messungen der meisten Objekte, Prozesse oder Ereignisse auf der Erde erlauben, ist die Indikatorextraktion eine Möglichkeit, diese Daten in aussagekräftige Informationen zu übersetzen.

Diese Arbeit gibt einen Überblick über den aktuellen Stand der kostenlosen und frei zugänglichen optischen EO-Daten in einem größeren Rahmen internationaler Initiativen wie den \acs{UN} \acp{SDG}, \emph{big Earth data} und \emph{data cube} Technologien. Basierend auf diesen Informationen wurde eine Proof-of-Concept-Implementierung eines semantischen \acs{EO} \emph{data cubes} erstellt, der täglich automatisch mit den neuesten Sentinel-2-Daten aktualisiert wird, einschließlich generischer semantischer Anreicherung. Es enthält kontinuierlich alle verfügbaren Daten für Sentinel-2 Granules, die sich im Nordwesten Syriens befinden, einem Land, in dem das ganze Jahr hindurch nur eine geringe monatliche Wolkenbedeckung gibt. Damit ist es ein idealer Standort für die Analyse von Oberflächen mit optischen multispektralen Daten, wie Sentinel-2. Die Ausgabe von semantischen Abfragen, die auf vegetations- und wasserähnlichen Semi-Konzepten aus dem generischen semantischen Informationslayer basieren, wird vorgestellt, einschließlich der Herausforderungen, denen sie bei ihrer Interpretation gegenüberstehen. Diese Ergebnisse könnten als Dokumentation der Oberflächenwasser- und Vegetationsdynamik im Laufe der Zeit gesehen werden. Ihr beabsichtigter Zweck ist es, ad-hoc reproduzierbare, wiederholbare und räumlich-explizite Informationen als Indikatoren zu generieren.


\end{otherlanguage}

\endgroup

\vfill
