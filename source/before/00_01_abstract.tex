%*******************************************************
% Abstract
%*******************************************************
%\renewcommand{\abstractname}{Abstract}
\cleardoublepage
\pdfbookmark[1]{Abstract}{Abstract}
\begingroup
%\let\clearpage\relax
\let\cleardoublepage\relax
\let\cleardoublepage\relax

\chapter*{Abstract}

Free and open-access \acf{EO} data with global coverage are being increasingly generated at higher spatial resolutions and temporal frequencies than ever before. Imagery with high spatial resolution (5-30\acs{m}) enables multi-temporal and persistent monitoring of large spatial extents for various applications. It does this in an unbiased, consistent way, independent of political borders, even if it is not always independent from political processes. Free and open \acs{EO} is a growing collection of data that is one of the few globally consistent sources available for generating information in support of international initiatives. However, this data requires automated workflows for handling, processing and analysis, including methods to convert data into valid information. Optical \acs{EO} data can not directly measure most objects, processes or events on Earth (e.g. digital numbers contain no semantics and many different surfaces can be represented by similar values). Indicator extraction is one way to translate this data into meaningful information to support actions towards meeting global agreements.

This thesis presents an overview of the current state of open and free \acs{EO} data in a larger framework of international initiatives, such as the \acs{UN} \acp{SDG}, big Earth data and data cube technologies. Based on this information, a proof-of-concept implementation of a semantic \acs{EO} data cube was created that is automatically updated with the newest Sentinel-2 data on a daily basis, including automatically generated generic semantic enrichment. It continuously incorporates all of the available Sentinel-2 data covering the study area located in north-western Syria (30,000\acs{km}²), which, at the time of writing, encompasses nearly 600 scenes.

This work demonstrates one example of how free and open \acs{EO} data can be efficiently and automatically used to generate information products. The output of semantic queries based on vegetation- and water-like semi-concepts from the generic semantic information layers are presented, including challenges faced in their interpretation. These outputs could be seen as documenting surface water and vegetation dynamics over time. Their intended purpose is to move in the direction of generating ad-hoc reproducible, scalable, repeatable and spatially-explicit information as indicators to support specific environment related targets of the \acp{SDG}.


\emph{Keywords:} remote sensing | big Earth data | data cube | semantic enrichment | reproducible research | sustainable development goals


\vfill



\begin{otherlanguage}{ngerman}
\clearpage
\pdfbookmark[1]{Zusammenfassung}{Zusammenfassung}
\chapter*{Zusammenfassung}

\small
Kostenlose und frei zugängliche Erdbeobachtungsdaten (engl. Earth observation, \acs{EO}) mit globaler Abdeckung werden zunehmend mit höherer räumlicher Auflösungen und zeitlichen Wiederholungsraten als je zuvor aufgenommen. Bilder mit hoher räumlicher Auflösung (5-30\acs{m}) ermöglichen die multi-temporale und dauerhafte Überwachung großer räumlicher Ausdehnung für unterschiedliche Anwendungen. Dies geschieht auf eine unvoreingenommene, konsistente Weise, unabhängig von politischen Grenzen aber nicht unbedingt unabhängig von politischen Entscheidungsprozessen. Die Erdbeobachtung basiert auf einer wachsenden Sammlung von Daten, die eine der wenigen beständigen Quellen globaler Abdeckung sind, und kann damit zur Generierung von Informationen zur Unterstützung internationaler Initiativen genutzt werden. Diese Daten erfordern jedoch automatisierte Workflows zur Handhabung, Verarbeitung und Analyse, einschließlich Methoden um aus den Daten Informationen zu produzieren. Da solche optischen \acs{EO}-Daten keine direkten Messungen der meisten Objekte, Prozesse oder Ereignisse auf der Erde erlauben (z.B. Digitalzahlen von \acs{EO}-Daten haben keine Semantik), ist die Indikatorextraktion eine Möglichkeit, diese Daten in aussagekräftige Informationen zu übersetzen (z.B. für internationale Initiative).

Diese Arbeit gibt einen Überblick über den aktuellen Stand der kostenlosen und frei zugänglichen optischen \acs{EO}-Daten in einem größeren Rahmen internationaler Initiativen wie den Nachhaltigen Entwicklungszielen (engl. Sustainable Development Goals, \acp{SDG}) der Vereinten Nationen (engl. United Nations, \acs{UN}), \emph{big Earth data} und \emph{data cube} Technologien. Basierend auf diesen Informationen wurde eine Proof-of-Concept-Implementierung eines semantischen \acs{EO} \emph{data cubes} erstellt, der täglich automatisch mit den neuesten Sentinel-2-Daten aktualisiert wird, einschließlich generischer semantischer Anreicherung der Daten. Der data cube enthält mit kontinuierlicher Aktualisierung alle verfügbaren Sentinel-2 Daten aber auch automatisch generierten generischen semantischen Informationsschichten, die sich in der Untersuchungsgebiet im Nordwesten Syriens befinden (30,000\acs{km}²). Die Ergebnisse semantischer Abfragen, die auf vegetations- und wasserähnlichen Semi-Konzepten aus dem generischen semantischen Informationsschichten basieren, wird vorgestellt, einschließlich der Herausforderungen, denen sie bei ihrer Interpretation gegenüberstehen. Die Arbeit zeigt am Beispiel von Sentinel-2, wie Erdbeobachungsdaten effizient ausgewertet werden können, um sie zur Erstellung von Informationsprodukten im Kontext der \acp{SDG} zu nutzen. Dies wurde beispielhaft für Oberflächenwasser und Vegetationsdynamik mit semantischen Abfragen umgesetzt. Ihr beabsichtigter Zweck ist es, ad-hoc reproduzierbare, räumlich skalierbare, wiederholbare und räumlich-explizite Informationen als Indikatoren zu generieren.

\normalsize


\end{otherlanguage}

\endgroup

\vfill
